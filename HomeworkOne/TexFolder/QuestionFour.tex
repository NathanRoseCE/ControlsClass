\begin{document}
\item Find the transfer function matrix for the system described as
  \begin{equation}
\dot x = \begin{bmatrix}
-1 & -2 & 6\\
2 & -3 & -2\\
-2 & -2 & 1
\end{bmatrix}
x + \begin{bmatrix}
1 & 1\\
1 & -1\\
1 & 0
\end{bmatrix}
u\end{equation}
\begin{equation}
y = \begin{bmatrix}
-1 & -1 & 2\\
1 & 1 & -1
\end{bmatrix}
x + \begin{bmatrix}
0 & 0\\
0 & 0
\end{bmatrix}
u\end{equation}

  
  \begin{equation}
    F(s) = C(sI-A)B + D
  \end{equation}
  That is the equation for how to do that and I could grind through this process but honestly that sounds
  terrible and ill probably make a bunch of minor math mistakes. So I am going to put this in python using thier
  control library and spit out an answer.

  Or I was until I heard you mention no ss2tf use, only basic matrix operations. I implimented it using
  sympy which allowed me to put the equation in and tell it to simplify. 

  \begin{equation}
\left[\begin{matrix}- \frac{6 s + 30}{s^{3} + 3 s^{2} + 11 s + 57} & - \frac{6 s + 18}{s^{3} + 3 s^{2} + 11 s + 57}\\\frac{s^{2} + 6 s + 29}{s^{3} + 3 s^{2} + 11 s + 57} & \frac{6 \left(s + 1\right)}{s^{3} + 3 s^{2} + 11 s + 57}\end{matrix}\right]\end{equation}

  
\end{document}