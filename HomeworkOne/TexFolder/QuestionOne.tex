\begin{document}
\item Consider the following system consisting of a cart of mass m attached to a rigid wall by a spring and a
damper. The spring stiffness $k(w)$ is a nonlinear function of displacement $w$ such that the spring pull(force)
applied to the mass is in the form of $f_s = k_1w - k_3w^3$, where $k_1$ and $k_3$ are constant scalars. Assume
that the cart rolls freely without friction. It can be shown that the motion is
$m\ddot w + c\dot w + k_1w - k_3w^3 = u$
\begin{enumerate}
\item The cart is instrumented with an accelerometer, which provides a measurement equation of the form
  $y = \ddot w$. Express the nonlinear equations of motion, including the output equation in state space form\\
  This system has two state variables: position and velocity. In order to keep it consistent with the notation
  in question \ref{one_equilibrium}. the following will be used.
  \begin{equation}
    x =
    \begin{bmatrix}
      position \\
      velocity
    \end{bmatrix} =
    \begin{bmatrix}
      w \\
      \dot w
    \end{bmatrix}
  \end{equation}
  \begin{equation}
    k_1w = u + k_3w^3 - c \dot w - m \ddot w
  \end{equation}
  \begin{equation}
    w = \frac{u + k_3w^3 - c \dot w - m \ddot w}{k_1}
  \end{equation}
  \begin{equation}
    c\dot w = u + k_3w^3 - k_1w - m \ddot w
  \end{equation}
  \begin{equation}
    \dot w = \frac{u + k_3w^3 - k_1w - m \ddot w}{c}
  \end{equation}
\item Assume $m = 1$, $k_1 = 4$,
  $k_3 = 1$, $c = 1$. Determine all the
  equilibrium points of the system assuming $u = 0$ \\
  {\LARGE \color{red} TODO: Figure out how to find equilibrium points}
\item Using the numerical assumptions above, linearize the equations of motion about the equilibrium point
  \label{one_equilibrium}
  \begin{equation}
    \begin{bmatrix}
2\\
0
\end{bmatrix}
\end{equation}
  That is provide the lienar state space model about the point in the form of:
  \begin{center}
    \begin{tabular}{r c l}
      $\delta \dot{\bar x}$ & $=$ & $A\delta\bar{x}(t) + B\delta\bar{u}(t)$\\
      $\delta y$ & $=$ & $C\delta\bar x(t) + D\delta\bar u(t)$\\
      $\delta \bar x$ & $=$ & $C\delta\bar x(t)$
    \end{tabular}
  \end{center}
  For the system that is described by:
  \begin{equation}
    w = \frac{0 + 1w^3 - 1\dot w - 1\ddot w}{4}
  \end{equation}
  \begin{equation}
    \dot w = \frac{0 + 1w^3 - 4w- 1\ddot w}{1}
  \end{equation}
  It can be linearized into:
  \begin{equation}
    w = 1w + 2\dot w + 5u
  \end{equation}
  \begin{equation}
    w = 3w + 4\dot w + 6u
  \end{equation}
  which in state space can be represented as:
  \begin{equation}
\dot x = \begin{bmatrix}
1 & 2\\
3 & 4
\end{bmatrix}
x + \begin{bmatrix}
5\\
6
\end{bmatrix}
u
\end{equation}
\begin{equation}
y = \begin{bmatrix}
1 & 0\\
0 & 1
\end{bmatrix}
x + \begin{bmatrix}
0\\
0
\end{bmatrix}
u
\end{equation}

\end{enumerate}
\end{document}