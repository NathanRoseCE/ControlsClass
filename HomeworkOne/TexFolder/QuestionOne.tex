\begin{document}
\item Consider the following system consisting of a cart of mass m attached to a rigid wall by a spring and a
damper. The spring stiffness $k(w)$ is a nonlinear function of displacement $w$ such that the spring pull(force)
applied to the mass is in the form of $f_s = k_1w - k_3w^3$, where $k_1$ and $k_3$ are constant scalars. Assume
that the cart rolls freely without friction. It can be shown that the motion is
$m\ddot w + c\dot w + k_1w - k_3w^3 = u$
\begin{enumerate}
\item The cart is instrumented with an accelerometer, which provides a measurement equation of the form
  $y = \ddot w$. Express the nonlinear equations of motion, including the output equation in state space form\\
  This system has two state variables: position and velocity. In order to keep it consistent with the notation
  in question \ref{one_equilibrium}. the following will be used.
  \begin{equation}
    x =
    \begin{bmatrix}
      position \\
      velocity
    \end{bmatrix} =
    \begin{bmatrix}
      w \\
      \dot w
    \end{bmatrix}
  \end{equation}
  \begin{equation}
    \ddot w = \frac{u - c\dot w - k_1w + k_3w^3}{m}
  \end{equation}
  \begin{equation}
    c\dot w = u + k_3w^3 - k_1w - m \ddot w
  \end{equation}
  \begin{equation}
    \dot w = \frac{u + k_3w^3 - k_1w - m \ddot w}{c}
  \end{equation}
\item Assume $m = 1$, $k_1 = 4$,
  $k_3 = 1$, $c = 1$. Determine all the
  equilibrium points of the system assuming $u = 0$ \\
  \begin{equation}
    \begin{bmatrix}
      \frac{u + k_3w^3 - k_1w - m \ddot w}{c} \\
      \frac{u - c\dot w - k_1w + k_3w^3}{m}
    \end{bmatrix} =
    \begin{bmatrix}
      0 \\
      0
    \end{bmatrix}
  \end{equation}
  \begin{equation}
    \begin{bmatrix}
      \frac{1w^3 - 4w -  1 \ddot w}{c} \\
      \frac{- c\dot w - 4w + 1w^3}{1}
    \end{bmatrix} =
    \begin{bmatrix}
      0 \\
      0
    \end{bmatrix}
  \end{equation}

  \begin{equation}
    \begin{bmatrix}
      \dot x_0 \\
      \dot x_1
    \end{bmatrix} =
    \begin{bmatrix}
      x_1 \\
      \frac{u - c\dot w - k_1w + k_3w^3}  m
    \end{bmatrix}
  \end{equation}
  because u is 0, that term can be discarded. then we can find in what states $A=0$
  
  \begin{equation}
    \begin{bmatrix}
      0 \\
      0
    \end{bmatrix} =
    \begin{bmatrix}
      x_1 \\
      \frac{-c\dot w - k_1w + k_3w^3} m
    \end{bmatrix}
  \end{equation}
  obviously $x_1$ must always be 0
  \begin{equation}
    0 =
    \frac{-c\dot w - k_1w + k_3w^3} m =
    -c\dot w - k_1w + k_3w^3
  \end{equation}
  which allows us to simplify
  \begin{equation}
    0 =
    -c\dot w - k_1w + k_3w^3=
    k_1x_0 + k_3{x_0}^3
  \end{equation}
  \begin{equation}
    0 =
    -k_1x_0 + k_3{x_0}^3 =
    x_0(-k_1 + k_3{x_0}^2)
  \end{equation}
  one solution for $x_0$ is 0
  \begin{equation}
    0 = -k_1 + k_3{x_0}^2
  \end{equation}
  \begin{equation}
    x_0 = \sqrt{frac{k_1}{k_3}} = \sqrt{4} = 2
  \end{equation}
  two equilibrium points are:
  \begin{equation}
    x_{eq} = 
    \begin{bmatrix}
      0 \\
      0
    \end{bmatrix},
    \begin{bmatrix}
      2\\
      0
    \end{bmatrix}
  \end{equation}
\item Using the numerical assumptions above, linearize the equations of motion about the equilibrium point
  \label{one_equilibrium}
  \begin{equation}
\begin{bmatrix}
2\\
0
\end{bmatrix}\end{equation}

  That is provide the linear state space model about the point in the form of:
  \begin{center}
    \begin{tabular}{r c l}
      $\delta \dot{\bar x}$ & $=$ & $A\delta\bar{x}(t) + B\delta\bar{u}(t)$\\
      $\delta y$ & $=$ & $C\delta\bar x(t) + D\delta\bar u(t)$\\
      $\delta \bar x$ & $=$ & $C\delta\bar x(t)$
    \end{tabular}
  \end{center}
  For the system that is described by:
  \begin{equation}
    \dot w = \dot w
  \end{equation}
  \begin{equation}
    \ddot w = \frac{0 + 1w^3 - 4w- 1\ddot w}{1}
  \end{equation}

  It can be linearized as seen above
  into:
  \begin{equation}
    w = 0w + 1\dot w + 0u
  \end{equation}
  \begin{equation}
    w = 8w + -1.0\dot w + 1.0u
  \end{equation}
  which in state space can be represented as:
  \begin{equation}
\dot x = \begin{bmatrix}
0 & 1\\
8 & -1.0
\end{bmatrix}
x + \begin{bmatrix}
0\\
1.0
\end{bmatrix}
u\end{equation}
\begin{equation}
y = \begin{bmatrix}
1 & 0\\
0 & 1
\end{bmatrix}
x + \begin{bmatrix}
0\\
0
\end{bmatrix}
u\end{equation}

\end{enumerate}
\end{document}