\item For each of the systems (a) and (b) do the following:
  \begin{enumerate}
  \item Obtain the eigenvalues \\
    \begin{enumerate}
    \item A\\
      \begin{equation}
        0 =
        \Delta(\lambda) =
        \vert \lambda I - A\vert =
        \begin{vmatrix}
          \lambda + 6 & 1 & 0 \\
          11 & \lambda & 1 \\
          6 & 0 & \lambda \\
        \end{vmatrix}
      \end{equation}
      \begin{equation}
        0 =
        (\lambda + 6)(\lambda^2) + 11\lambda + 6 =
        \lambda^3 + 6\lambda^2 + 11\lambda + 6 =
      \end{equation}
      \begin{equation}
        0 =
        \lambda^3 + 6\lambda^2 + 11\lambda + 6 =
        (x + 1)(x^2 + 5x + 6) = 
        (x + 1)(x + 2)(x + 3)
      \end{equation}
      \begin{equation}
        \lambda = -1, -2, -3
      \end{equation}
      \begin{equation}
        0 =
        \vert \lambda I - A\vert =
        \begin{bmatrix}
          \lambda + 6 & 1 & 0 \\
          11 & \lambda & 1 \\
          6 & 0 & \lambda \\
        \end{bmatrix}v
      \end{equation}
      \begin{equation}
        0 = 
        \begin{bmatrix}
          5 & 1 & 0 \\
          11 & -1 & 1 \\
          6 & 0 & -1 \\
        \end{bmatrix}v_1
      \end{equation}
      \begin{equation}
        v_1 =
        \begin{bmatrix}
          \frac 1 2 \\
          \frac 3 2 \\
          1
        \end{bmatrix}
      \end{equation}
      \begin{equation}
        0 = 
        \begin{bmatrix}
          4 & 1 & 0 \\
          11 & -2 & 1 \\
          6 & 0 & -2 \\
        \end{bmatrix}v_1
      \end{equation}
      \begin{equation}
        v_2 =
        \begin{bmatrix}
          \frac 1 3 \\
          \frac 4 3 \\
          1
        \end{bmatrix}
      \end{equation}
      \begin{equation}
        0 = 
        \begin{bmatrix}
          3 & 1 & 0 \\
          11 & -3 & 1 \\
          6 & 0 & -3 \\
        \end{bmatrix}v_1
      \end{equation}
      \begin{equation}
        v_3 =
        \begin{bmatrix}
          \frac 1 6 \\
          \frac 5 6 \\
          1
        \end{bmatrix}
      \end{equation}
      \begin{equation}
        V =
        \begin{bmatrix}
          \frac 1 2 & \frac 1 3 & \frac 1 6
          \frac 3 2 & \frac 4 3 & \frac 5 6\\
          1 & 1 & 1
        \end{bmatrix}
      \end{equation}
    \item B\\
      eigenvalues are: $\lambda = -1, -1, -1, -2, -3, -3$
      from sympy.jordan\_from, eigenvectors are:
      \begin{equation}
        V = 
        \left[\begin{matrix}0 & 0 & 0 & 1 & 0 & 0\\0 & 0 & 1 & 0 & 1 & 0\\0 & 0 & 0 & 0 & 1 & 0\\1 & 0 & 0 & 0 & 0 & 0\\0 & 1 & 0 & 0 & 0 & \frac{1}{2}\\0 & 0 & 0 & 0 & 0 & 1\end{matrix}\right]
      \end{equation}
    \end{enumerate}
  \item Use the similarity transformation $x(t) = Vz(t)$ to express the system in the modal form\\
    \begin{enumerate}
    \item A\\
      \begin{equation}
        \bar A = J =
        \begin{bmatrix}
          -3 & 0 & 0 \\
          0 & -2 & 0 \\
          0 & 0 & -1 \\
        \end{bmatrix}
      \end{equation}
      \begin{equation}
        \bar B = V^{-1}B = 
        \begin{bmatrix}
          -4 \\
          9 \\
          0
        \end{bmatrix}
      \end{equation}
      \begin{equation}
        \bar C = CV = 
        \begin{bmatrix}
          \frac 1 2 & \frac 1 3 & \frac 1 6
        \end{bmatrix}
      \end{equation}
    \item B\\
      jordan form:
      \begin{equation}
        \bar A = J = 
        \left[\begin{matrix}-3 & 0 & 0 & 0 & 0 & 0\\0 & -3 & 0 & 0 & 0 & 0\\0 & 0 & -2 & 0 & 0 & 0\\0 & 0 & 0 & -1 & 0 & 0\\0 & 0 & 0 & 0 & -1 & 0\\0 & 0 & 0 & 0 & 0 & -1\end{matrix}\right]
      \end{equation}
      \begin{equation}
        \bar B = V^{-1}B = 
        \left[\begin{matrix}0 & 1\\- \frac{1}{2} & 0\\0 & -1\\1 & 0\\0 & 1\\1 & 0\end{matrix}\right]
      \end{equation}
      \begin{equation}
        \bar C = CV =
        \left[\begin{matrix}0 & 0 & 2 & 1 & 2 & 0\\1 & 1 & 0 & 0 & 0 & \frac{1}{2}\end{matrix}\right]
      \end{equation}
    \end{enumerate}
  \item Use the modal form to study the controllability and observability of the system(slides 77, 85)\\
    \begin{enumerate}
    \item A\\
      modal state variables 1 and 2 are controllable do to non-zero values but 3 is not. all state variables are
      observable due to non-zero values. 
    \item B\\
      state variables are 1-3 are controllable, 4-6 are not due to non-linear independence of bottom three rows
      of B. only the -2 eigenvalue state variable is observable because of its linear independence 
    \end{enumerate}
  \item idicate teh controllability and onbservability of each mode\\
    \begin{enumerate}
    \item A\\
      System is observable not controllable
    \item B\\
      system is not controllobale or observable
    \end{enumerate}
  \item Study stabalizability and detectability of the system\\
    \begin{enumerate}
    \item A\\
      system is stabalizable(and obviously) detectable
    \item B\\
      system is stabalizable and detectable
    \end{enumerate}
  \item Plot the block diagram of the modal form\\
    \begin{enumerate}
    \item A\\
      {\LARGE \color{red} TODO: this}
    \item B\\
      {\LARGE \color{red} TODO: this}
    \end{enumerate}
  \end{enumerate}