\item For the systems of a and b do the following:
  \begin{enumerate}
  \item Obtain the transfer function matrix \\
    \begin{enumerate}
    \item A\\
      \begin{equation}
        G(s) =
        C(sI-A)^{-1}B + D =
        \begin{bmatrix}
          3 & 1 & 4 & 0 \\
          0 & 1 & 0 & 1 
        \end{bmatrix}
        \begin{bmatrix}
          s + 6 & -1 & 0 & 0 \\
          0 & s + 6 & 0 & 0 \\
          0 & 0 & s + 6 & 0 \\
          0 & 0 & 0 & s - 6
        \end{bmatrix}^{-1}
        \begin{bmatrix}
          0 & 0 & 0 \\
          1 & 1 & 1 \\
          2 & 0 & 2 \\
          0 & 1 & 0
        \end{bmatrix}
      \end{equation}
      \begin{equation}
        G(s) =
        \frac 1 {(s+6)^2(s-6)}
        \begin{bmatrix}
          3 & 1 & 4 & 0 \\
          0 & 1 & 0 & 1 
        \end{bmatrix}
        \begin{bmatrix}
          s^2 - 36 & s - 6 & 0 & 0 \\
          0 & s^2 - 36 & 0 & 0 \\
          0 & 0 & s^2 - 36 & 0 \\
          0 & 0 & 0 & (s + 6)^2
        \end{bmatrix}
        \begin{bmatrix}
          0 & 0 & 0 \\
          1 & 1 & 1 \\
          2 & 0 & 2 \\
          0 & 1 & 0
        \end{bmatrix}
      \end{equation}
      \begin{equation}
        G(s) =
        \frac 1 {(s+6)^2(s-6)}
        \begin{bmatrix}
          3 & 1 & 4 & 0 \\
          0 & 1 & 0 & 1 
        \end{bmatrix}
        \begin{bmatrix}
          s - 6 & s - 6 & s - 6\\
          s^2 - 36  & s^2 - 36 & s^2 - 36 \\
          2s^2 - 72  & 0 & 2s^2 - 72 \\
          0 & (s + 6)^2 & 0
        \end{bmatrix}
      \end{equation}
      \begin{equation}
        G(s) =
        \frac 1 {(s+6)^2(s-6)}
        \begin{bmatrix}
          9s^2 + 3s - 342 & s^2 + 3s - 54 & 9s^2 + 3s - 342 \\
          s^2 - 36 & 2s(s+6) & s^2 - 36 \\
        \end{bmatrix}
      \end{equation}
    \item B\\
      from previous problem, only thing that changes is C
      \begin{equation}
        G(s) =
        \frac 1 {(s+6)^2(s-6)}
        \begin{bmatrix}
          3 & 1 & 4 & 0 \\
          0 & 1 & 1 & 1 
        \end{bmatrix}
        \begin{bmatrix}
          s - 6 & s - 6 & s - 6\\
          s^2 - 36  & s^2 - 36 & s^2 - 36 \\
          4s^2 - 144  & 0 & 4s^2 - 144 \\
          0 & (s + 6)^2 & 0
        \end{bmatrix}
      \end{equation}
      \begin{equation}
        G(s) =
        \frac 1 {(s+6)^2(s-6)}
        \begin{bmatrix}
          9s^2 + 3s - 342 & s^2 + 3s - 54 & 9s^2 + 3s - 342 \\
          s^2 - 36 & 2s(s+6) & s^2 - 36 \\
        \end{bmatrix}
      \end{equation}
    \end{enumerate}
  \item Verify the system is a minimum realization, if the system is not minimal do steps 3 and 4 \\
    A minimum realization is both observable and controllable
    \begin{enumerate}
    \item A\\
      Controllable because rows 2 and 3 of matrix B are LI, and row 4 is non-zero
      not observable because columns 1 and 3 of matrix C are not LI
      This is a not a min realization
    \item B\\
      Controllable because rows 2 and 3 of matrix B are LI, and row 4 is non-zero
      Observable because columns 1 and 3 of matrix C are LI and column 4 is non-zero
      This is a min realization
    \end{enumerate}
  \item Use approach a discussed in slide 92 to obtain the minimal realization of the system. \\
    \begin{enumerate}
    \item A\\
      Move to observable canonical form
      $q = 2, r =3$
      \begin{equation}
        (s+6)^2(s-6) = (s^2 + 12s + 36)(s-6) = s^3 + 6s^2 - 36s - 216
      \end{equation}
      \begin{equation}
        N_0 =
        \begin{bmatrix}
          0 & 0 & 0 \\
          0 & 0 & 0 \\
        \end{bmatrix}
      \end{equation}
      \begin{equation}
        N_1 =
        \begin{bmatrix}
          9 & 1 & 9 \\
          1 & 2 & 1 \\
        \end{bmatrix}
      \end{equation}
      \begin{equation}
        N_2 =
        \begin{bmatrix}
          3 & 3 & 3 \\
          0 & 12 & 0 \\
        \end{bmatrix}
      \end{equation}
      \begin{equation}
        N_3 =
        \begin{bmatrix}
          -342 & -54 & -342 \\
          -36 & 12 & -36 \\
        \end{bmatrix}
      \end{equation}
      \begin{equation}
        A =
        \begin{bmatrix}
          0 & 0 & 0 & 0 & 216 & 0 \\
          0 & 0 & 0 & 0 & 0 & 216 \\
          1 & 0 & 0 & 0 & 36 & 0 \\
          0 & 1 & 0 & 0 & 0 & 36 \\
          0 & 0 & 1 & 0 & -6 & 0 \\
          0 & 0 & 0 & 1 & 0 & -6 \\
        \end{bmatrix}
      \end{equation}
      \begin{equation}
        B =
        \begin{bmatrix}
          -342 & -54 & -342 \\
          -36 & 12 & -36 \\
          3 & 3 & 3 \\
          0 & 12 & 0 \\
          9 & 1 & 9 \\
          1 & 2 & 1 \\
        \end{bmatrix}
      \end{equation}
      \begin{equation}
        C =
        \begin{bmatrix}
          0 & 0 & 0 & 0 & 1 & 0 \\
          0 & 0 & 0 & 0 & 0 & 1 \\
        \end{bmatrix}
      \end{equation}
      \begin{equation}
        D =
        \begin{bmatrix}
          0 & 0 & 0 \\
          0 & 0 & 0 \\
        \end{bmatrix}
      \end{equation}
      Now move it to jordan form:
      \begin{equation}
        P =
        \left[\begin{matrix}-36 & -6 & 0 & 0 & 36 & 0\\0 & 0 & -36 & -6 & 0 & 36\\0 & 1 & 0 & 0 & 12 & 0\\0 & 0 & 0 & 1 & 0 & 12\\1 & 0 & 0 & 0 & 1 & 0\\0 & 0 & 1 & 0 & 0 & 1\end{matrix}\right]
      \end{equation}
      \begin{equation}
        \bar A = J =
        \left[\begin{matrix}-6 & 1 & 0 & 0 & 0 & 0\\0 & -6 & 0 & 0 & 0 & 0\\0 & 0 & -6 & 1 & 0 & 0\\0 & 0 & 0 & -6 & 0 & 0\\0 & 0 & 0 & 0 & 6 & 0\\0 & 0 & 0 & 0 & 0 & 6\end{matrix}\right]
      \end{equation}
      \begin{equation}
        \bar B = P^{-1}B = 
        \left[\begin{matrix}-36 & -6 & 0 & 0 & 36 & 0\\0 & 0 & -36 & -6 & 0 & 36\\0 & 1 & 0 & 0 & 12 & 0\\0 & 0 & 0 & 1 & 0 & 12\\1 & 0 & 0 & 0 & 1 & 0\\0 & 0 & 1 & 0 & 0 & 1\end{matrix}\right]
        \begin{bmatrix}
          -342 & -54 & -342 \\
          -36 & 12 & -36 \\
          3 & 3 & 3 \\
          0 & 12 & 0 \\
          9 & 1 & 9 \\
          1 & 2 & 1 \\
        \end{bmatrix} =
        \left[\begin{matrix}9 & 1 & 9\\3 & 3 & 3\\1 & \frac{11}{12} & 1\\0 & -1 & 0\\0 & 0 & 0\\0 & \frac{13}{12} & 0\end{matrix}\right]
      \end{equation}
      \begin{equation}
        \bar C = CP = 
        \begin{bmatrix}
          0 & 0 & 0 & 0 & 1 & 0 \\
          0 & 0 & 0 & 0 & 0 & 1 \\
        \end{bmatrix}
        \left[\begin{matrix}-36 & -6 & 0 & 0 & 36 & 0\\0 & 0 & -36 & -6 & 0 & 36\\0 & 1 & 0 & 0 & 12 & 0\\0 & 0 & 0 & 1 & 0 & 12\\1 & 0 & 0 & 0 & 1 & 0\\0 & 0 & 1 & 0 & 0 & 1\end{matrix}\right] =
        \left[\begin{matrix}1 & 0 & 0 & 0 & 1 & 0\\0 & 0 & 1 & 0 & 0 & 1\end{matrix}\right]
      \end{equation}
      Painfully remove a state variable
      \begin{equation}
        y(t) =
        \begin{bmatrix}
          1 \\
          0
        \end{bmatrix}\bar x_0 + 
        \begin{bmatrix}
          0 \\
          0
        \end{bmatrix}\bar x_1 + 
        \begin{bmatrix}
          0 \\
          1
        \end{bmatrix}\bar x_2 + 
        \begin{bmatrix}
          0 \\
          0
        \end{bmatrix}\bar x_3 + 
        \begin{bmatrix}
          1 \\
          0
        \end{bmatrix}\bar x_4 + 
        \begin{bmatrix}
          0 \\
          1
        \end{bmatrix}\bar x_5
      \end{equation}
      Lets define the state variables as follows:
      \begin{enumerate}
      \item \phantom{.} $q_o = \bar x_0 + \bar x_4$
      \item \phantom{.} $q_1 = \bar x_1$
      \item \phantom{.} $q_2 = \bar x_2$
      \item \phantom{.} $q_3 = \bar x_3$
      \item \phantom{.} $q_4 = \bar x_5$
      \end{enumerate}
      Thus:
      \begin{enumerate}
      \item \phantom{.} $\dot q_o = \dot \bar x_0 + \dot \bar x_4$
      \item \phantom{.} $\dot q_1 = \dot \bar x_1$
      \item \phantom{.} $\dot q_2 = \dot \bar x_2$
      \item \phantom{.} $\dot q_3 = \dot \bar x_3$
      \item \phantom{.} $\dot q_4 = \dot \bar x_5$
      \end{enumerate}
      THe state update equation is the same as bar A but:\\
      \begin{enumerate}
      \item row 0 = row 0 + row 4
      \end{enumerate}
      for the output equation you just delete column 4 resulting in the following stat equation
      same thing, but swap row for column for C
      \begin{equation}
        \dot q =
        \left[\begin{matrix}-6 & 1 & 0 & 0 & 0\\0 & -6 & 0 & 0 & 0\\0 & 0 & -6 & 1 & 0\\0 & 0 & 0 & -6 & 0\\0 & 0 & 0 & 0 & 6\end{matrix}\right]q +
        \left[\begin{matrix}9 & 1 & 9\\3 & 3 & 3\\1 & \frac{11}{12} & 1\\0 & -1 & 0\\0 & \frac{13}{12} & 0\end{matrix}\right] u
      \end{equation}
      \begin{equation}
        y =
        \left[\begin{matrix}1 & 0 & 0 & 0 & 0\\0 & 0 & 1 & 0 & 1\end{matrix}\right] q +
        \left[\begin{matrix}0 & 0 & 0\\0 & 0 & 0\end{matrix}\right] u
      \end{equation}
    \item B\\
      not needed
    \end{enumerate}
  \item Show that the transfer matrix of the minimal realization is the same as that in part i \\
    \begin{enumerate}
    \item A\\
      \begin{equation}
        G(s) = C(sI-A)^{-1}B + D =
        \left[\begin{matrix}1 & 0 & 0 & 0 & 0\\0 & 0 & 1 & 0 & 1\end{matrix}\right]
        (\left[\begin{matrix}s + 6 & -1 & 0 & 0 & 0\\0 & s + 6 & 0 & 0 & 0\\0 & 0 & s + 6 & -1 & 0\\0 & 0 & 0 & s + 6 & 0\\0 & 0 & 0 & 0 & s - 6\end{matrix}\right])^{-1}
        \left[\begin{matrix}9 & 1 & 9\\3 & 3 & 3\\1 & \frac{11}{12} & 1\\0 & -1 & 0\\0 & \frac{13}{12} & 0\end{matrix}\right] + 0
      \end{equation}
      \begin{equation}
        G(s) = 
        \left[\begin{matrix}1 & 0 & 0 & 0 & 0\\0 & 0 & 1 & 0 & 1\end{matrix}\right]
        \left[\begin{matrix}\frac{1}{s + 6} & \frac{1}{s^{2} + 12 s + 36} & 0 & 0 & 0\\0 & \frac{1}{s + 6} & 0 & 0 & 0\\0 & 0 & \frac{1}{s + 6} & \frac{1}{s^{2} + 12 s + 36} & 0\\0 & 0 & 0 & \frac{1}{s + 6} & 0\\0 & 0 & 0 & 0 & \frac{1}{s - 6}\end{matrix}\right]
        \left[\begin{matrix}9 & 1 & 9\\3 & 3 & 3\\1 & \frac{11}{12} & 1\\0 & -1 & 0\\0 & \frac{13}{12} & 0\end{matrix}\right] + 0
      \end{equation}
      \begin{equation}
        G(s) =
        \left[\begin{matrix}\frac{3 \left(3 s + 19\right)}{s^{2} + 12 s + 36} & \frac{s + 9}{s^{2} + 12 s + 36} & \frac{3 \left(3 s + 19\right)}{s^{2} + 12 s + 36}\\\frac{1}{s + 6} & \frac{2 \left(s^{2} + 6 s + 6\right)}{s^{3} + 6 s^{2} - 36 s - 216} & \frac{1}{s + 6}\end{matrix}\right]
      \end{equation}
      \begin{equation}
        G(s) =
        \frac 1 {(s+6^2)(s-6)}
        \left[\begin{matrix}9 s^{2} + 3 s - 342 & s^{2} + 3 s - 54 & 9 s^{2} + 3 s - 342\\s^{2} - 36 & 2 s^{2} + 12 s + 12 & s^{2} - 36\end{matrix}\right]
      \end{equation}

    \item B\\
      not needed
    \end{enumerate}
  \end{enumerate}