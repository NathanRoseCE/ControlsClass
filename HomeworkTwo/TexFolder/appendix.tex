\appendix

% \section{Appendix}
\vspace{2cm}
\section{One A source code} \label{appendix:one-a-src}
\begin{minted}{python}
import numpy as np
import sympy
from sympy import eye, shape, simplify, inverse_laplace_transform, Matrix

def sI_A(A: Matrix):
    s = sympy.symbols('s')
    s_I = eye(shape(A)[0])*s
    return simplify(s_I-A)

def STM_laplace_inverse(A: Matrix):
    s, t = sympy.symbols('s, t')
    return simplify(
        inverse_laplace_transform((sI_A(A)).inv(), s, t)
    )
\end{minted}

\section{Two A source code} \label{appendix:two-a-src}
\begin{minted}{python}
from .one_a import *
from sympy import Matrix, simplify


def zero_input_equation(A: Matrix, x_0: Matrix):
    stm = STM_laplace_inverse(A) #from problem 1 
    return simplify(stm * x_0)


\end{minted}

\section{Two B source code} \label{appendix:two-b-src}
\begin{minted}{python}
from .one_a import *
from sympy import Matrix, simplify, exp, symbols, integrate

def get_integrand(A: Matrix, B: Matrix):
    stm = STM_laplace_inverse(A) # from problem 1
    t, tau = symbols('t, ' + r'\tau')
    u = exp(2*t)
    return simplify((stm.subs(t, t-tau) * B.subs(t, tau) * u.subs(t, tau)))
    
def zero_state(A: Matrix, B: Matrix):
    integrand = get_integrand(A,B)
    t, tau = symbols('t, ' + r'\tau')
    return simplify(integrate(integrand, (tau, 0, t)))
\end{minted}

\section{Three A source code} \label{appendix:two-b-src}
\begin{minted}{python}
from typing import Iterable, Tuple
from sympy import Matrix, latex, eye


def beta_equation_general() -> str:
    return r"f(\lambda) = \beta_0 + \beta_1\lambda + \beta_2\lambda^2"
                       
def beta_equation(eig_vals: list) -> Tuple[Matrix,Matrix]:
    """
    gets the matrix form of the equations above for a 3x3 and given eigenvalues.
    Assumes no multiplicity and f is inverse
    """
    values = []
    system = []
    for eig_val in eig_vals:
        values.append(
            pow(eig_val, -1)
        )
        system.append(
            [pow(eig_val, i) for i in range(len(eig_vals))]
        )
    return Matrix(values), Matrix(system)

def bmatrix() -> str:
    return r"\begin{bmatrix}\beta_0 \\ \beta_1 \\ \beta_2 \end{bmatrix}"

def beta_equation_str(eig_vals) -> str:
    values, system = beta_equation(eig_vals)
    return (f"{latex(values)} = {latex(system)}" + r"^{-1}" +
            bmatrix()
            )
def solved_beta_equation_str(eig_vals) -> str:
    values, system = beta_equation(eig_vals)
    solved = system.inv() * values
    return (
        bmatrix() + f" = {latex(solved)}" 
    )

def beta_values(eig_vals) -> Iterable[float]:
    values, system = beta_equation(eig_vals)
    solved = system.inv() * values
    return [solved[i, 0] for i in range(3)]

def final_answer(eig_vals, A) -> str:
    A = Matrix(A)
    I = eye(A.shape[0])
    beta_0, beta_1, beta_2 = beta_values(eig_vals)
    return (beta_0*I) + (beta_1*A) + (beta_2*A*A)
\end{minted}



Disclaimer: This is just a few relevant fragments of the source code, as the entire code is a complicated system
that takes these fragments and automatically renders them into the final pdf. However all of this is available
online on \href{https://github.com/NathanRoseCE/ControlsClass}{github}(its latex + python)