\item A system is described by
  \begin{equation}
\dot x = \begin{bmatrix}
-2 & 1\\
-1 & 0
\end{bmatrix}
x + \begin{bmatrix}
3\\
1
\end{bmatrix}
u\end{equation}

  Obtain the STM of the uncontrolled system using the following methods:
  \begin{enumerate}
  \item Via taking the laplace inverse of $(sI - A)^{-1}$ \\
    \begin{equation}
(sI-A) = \left[\begin{matrix}s + 2 & -1\\1 & s\end{matrix}\right]\end{equation}
\begin{equation}
(sI-A)^{-1} = \left[\begin{matrix}\frac{s}{s^{2} + 2 s + 1} & \frac{1}{s^{2} + 2 s + 1}\\- \frac{1}{s^{2} + 2 s + 1} & \frac{s + 2}{s^{2} + 2 s + 1}\end{matrix}\right]\end{equation}
\begin{equation}
\mathscr{L}^{-1}\Big ((sI-A)^{-1}\Big ) = \left[\begin{matrix}\left(1 - t\right) e^{- t} \theta\left(t\right) & t e^{- t} \theta\left(t\right)\\- t e^{- t} \theta\left(t\right) & \left(t + 1\right) e^{- t} \theta\left(t\right)\end{matrix}\right]\end{equation}

    
    The theta in the equation is the step function as described
    \href{https://math.stackexchange.com/questions/1967109/inverse-laplace-transfrom-using-sympy}{here}
    
    This is accomplished with the code in \autoref{appendix:one-a-src}
  \item Via model decomposition of Matrix A \\
    Eigenvalues are: $\lambda = -1, -1$
    This is not the simple case... joy
    \begin{equation}
      (\lambda_iI -A)v_i = 0
    \end{equation}
    \begin{equation}
      \begin{bmatrix}
        1 & -1 \\
        1 &  -1
      \end{bmatrix}v_i = 0
    \end{equation}
    only one solution :/

    
    \begin{equation}
      0 = 
      \begin{bmatrix}
        1 & -1 \\
        1 &  -1
      \end{bmatrix}^2v_2 =
      \begin{bmatrix}
        1 & 1 \\
        1 & 1
      \end{bmatrix}v_2
    \end{equation}
    \begin{equation}
      v_2 =
      \begin{bmatrix}
        1 \\
        -1
      \end{bmatrix}
    \end{equation}
    \begin{equation}
      v_1 =
      \begin{bmatrix}
        1 & -1 \\
        1 &  -1
      \end{bmatrix}v_2 = 
      \begin{bmatrix}
        1 & -1 \\
        1 &  -1
      \end{bmatrix}
      \begin{bmatrix}
        1 \\
        -1
      \end{bmatrix} = 
      \begin{bmatrix}
        2 \\
        2
      \end{bmatrix}
    \end{equation}
    \begin{equation}
      V =
      \begin{bmatrix}
        2 &  1 \\
        2 & -1
      \end{bmatrix},
      \Lambda =
      \begin{bmatrix}
        -1 &  1 \\
         0 & -1
      \end{bmatrix}
    \end{equation}
    {\LARGE \color{red} TODO: getting the wrong $V$ matrix}
  \item Via the Cayley-Hamilton theorem \\
    This will be done by using the Cayley-hamilton theorem to solve for $e^{At}$ then via equation 22 it is the
    STM
    \begin{itemize}
    \item Find the characteristic polynomial \\
      \begin{equation}
        0 = |\lambda I - A| =
        \begin{vmatrix}
          \lambda +2 & -1 \\
          1 & \lambda 
        \end{vmatrix} =
        \lambda^2 + 2\lambda + 1
      \end{equation}
      \begin{equation}
        \lambda = -1,-1
      \end{equation}
    \item solve for $\beta_0$ and $\beta_1$ \\
      \begin{equation}
        e^{-t} = \beta_0 - \beta_1
      \end{equation}
      \begin{equation}
        te^{-t} = \beta_1
      \end{equation}
      \begin{equation}
        e^{-t} = \beta_0 - te^{-t}
      \end{equation}
      \begin{equation}
        \beta_0 = e^{-t} + te^{-t}
      \end{equation}
    \item solve $e^{At}$(and STM)
      \begin{equation}
        STM = e^{At} = \beta_oI + \beta_1A =
        \begin{bmatrix}
          e^{-t} - te^{-t} & te^{-t}\\
          -te^{-t} & te^{-t} + e^{-t}
        \end{bmatrix}
      \end{equation}
    \end{itemize}
  \end{enumerate}