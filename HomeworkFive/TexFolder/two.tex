\item Condider a LTI MIMO system with matrices:
  \begin{equation}
    A =
    \begin{bmatrix}
      -2 & -2 & 0 \\
      0 & 0 & 1 \\
      0 & -3 & -4 
    \end{bmatrix}
    B =
    \begin{bmatrix}
      1 & 0 \\
      0 & 0 \\
      0 & 1
    \end{bmatrix}
  \end{equation}
  Assuming all stat variabes can be measured find a state feedback control law such that the eigenvalues of the
  closed loop system  become: $\lambda_{d_1} =\lambda_{d_2} = -3$ and $\lambda_{d_3} = -4$. For this purpose use:
  \begin{enumerate}
  \item Method 1 in slide 106 and verify the eigenvalues of the closed loop system \\
    \begin{enumerate}
    \item For each deisred eigenvalue $\lambda_{d_i}$: solve:
      \begin{equation}
        \begin{bmatrix}
          \lambda_{d_i}I - A & B
        \end{bmatrix}\bar \phi_i =
        \bar 0
      \end{equation}
      \begin{enumerate}
      \item 1, 2
        \begin{equation}
          \begin{bmatrix}
            -1 & 2 & 0 & 1 & 0\\
            0 & -3 & -1 & 0 & 0 \\
            0 & 3 & 1  & 0 & 1
          \end{bmatrix}\phi_i = \bar 0
        \end{equation}
        \begin{equation}
          \phi_1 =
          \begin{bmatrix}
            1 \\
            0 \\
            0 \\
            1 \\
            0 
          \end{bmatrix}
          \phi_2 =
          \begin{bmatrix}
            2 \\
            1 \\
            -3 \\
            0 \\
            0 
          \end{bmatrix}
        \end{equation}
      \item 3
        \begin{equation}
          \begin{bmatrix}
            -2 & 2 & 0 & 1 & 0\\
            0 & -4 & -1 & 0 & 0 \\
            0 & 3 & 0  & 0 & 1
          \end{bmatrix}\phi_3 = \bar 0
        \end{equation}
        \begin{equation}
          phi_3 =
          \begin{bmatrix}
            1 \\
            1 \\
            -4 \\
            0 \\
            -3
          \end{bmatrix}
        \end{equation}
      \item split the matrix
        \begin{equation}
          \begin{bmatrix}
            \bar \psi_1 & \bar \psi_2 & \bar \psi_3
          \end{bmatrix} =
          \begin{bmatrix}
            1 & 2 & 1 \\
            0 & 1 & 1 \\
            0 & -3 & -4
          \end{bmatrix}
        \end{equation}
        \begin{equation}
          \begin{bmatrix}
            K\bar \psi_1 & K\bar \psi_2 & K\bar \psi_3
          \end{bmatrix} = 
          \begin{bmatrix}
            1 & 0 & 0 \\
            0 & 0 & -3
          \end{bmatrix}
        \end{equation}
      \item Solve for K
        \begin{equation}
          K = 
          \begin{bmatrix}
            K\bar \psi_1 & K\bar \psi_2 & K\bar \psi_3
          \end{bmatrix}
          \begin{bmatrix}
            \bar \psi_1 & \bar \psi_2 & \bar \psi_3
          \end{bmatrix}^{-1} =
          \begin{bmatrix}
            0 & 2 & 2 \\
            -\frac 3 2 & -\frac 3 2 & 3
          \end{bmatrix}
          \begin{bmatrix}
            1 & 2 & 0 \\
            1 & 1 & -1 \\
            -\frac 3 2 & -\frac 3 2 & 2
          \end{bmatrix}^{-1}
        \end{equation}
        \begin{equation}
          K =
          \begin{bmatrix}
            0 & 2 & 2 \\
            -\frac 3 2 & -\frac 3 2 & 3
          \end{bmatrix}
          \begin{bmatrix}
            1 & 2 & 0 \\
            1 & 1 & -1 \\
            -\frac 3 2 & -\frac 3 2 & 2
          \end{bmatrix}^{-1}=
          \begin{bmatrix}
            1 & -5 & -1 \\
            0 & 9 & 3
          \end{bmatrix}
        \end{equation}
      \item evalute eigenvalues of feedback matrix
        \begin{equation}
          A - BK =
          \begin{bmatrix}
            -2 & -2 & 0 \\
            0 & 0 & 1 \\
            0 & -3 & -4 
          \end{bmatrix} -
          \begin{bmatrix}
            1 & 0 \\
            0 & 0 \\
            0 & 1
          \end{bmatrix}
          \begin{bmatrix}
            1 & -5 & -1 \\
            0 & 9 & 3
          \end{bmatrix} =
          \begin{bmatrix}
            -3 & 3 & 1 \\
            0 & 0 & 1 \\
            0 & -12 & -7
          \end{bmatrix}
        \end{equation}
        eigenvalues of this are: $-3, -3, -4$
      \end{enumerate}
    \end{enumerate}
  \item Method 2 in slide 107 and verify the eigenvalues of the closed loop system\\
    \begin{enumerate}
    \item get a $k_V$ where you have distict eigenvalues\\
      already done $k_v = 0$
    \item select a $\bar v$ such that $(A, b\bar v)$ is controllable:
      \begin{equation}
        \bar v =
        \begin{bmatrix}
          0 \\
          1
        \end{bmatrix}
      \end{equation}
      \begin{equation}
        Bv = \begin{bmatrix}
          0 \\
          0 \\
          1
        \end{bmatrix}
      \end{equation}
      \begin{equation}
        \begin{bmatrix}
          A^2Bv & ABv & Bv
        \end{bmatrix} =
        \begin{bmatrix}
          -2 & 0 & 0 \\
          -4 & 1 & 0 \\
          -13 & -4 & 1 \\
        \end{bmatrix}
      \end{equation}
      full row rank $\implies$ controllable
      \begin{equation}
        A =
        \begin{bmatrix}
          -2 & -2 & 0 \\
          0 & 0 & 1 \\
          0 & -3 & -4 
        \end{bmatrix}
        B =
        \begin{bmatrix}
          0 \\
          0 \\
          1
        \end{bmatrix}
      \end{equation}
      \begin{equation}
        \Delta_d(s) =
        (s+3)(s+3)(s+4) =
        (s^2 + 6s + 9)(s + 4)
      \end{equation}
      \begin{equation}
        \Delta_d(s) =
        s^3 + 10s^2 + 33s + 36
      \end{equation}
      \begin{equation}
        \bar \alpha =
        \begin{bmatrix}
          10 & 33 & 36
        \end{bmatrix}
      \end{equation}
      \begin{equation}
        \Delta(s) = \vert sI -A \vert =
        \begin{vmatrix}
          s + 2 & 2 & 0 \\
          0 & s & -1 \\
          0 & 3 & s + 4
        \end{vmatrix} =
        (s+2)( (s)(s+4) + 3 )
      \end{equation}
      \begin{equation}
        \Delta(s) = 
        (s+2)( s^s+4s + 3 ) =
        s^3 + 6s^2 + 11s + 6
      \end{equation}
      \begin{equation}
        \alpha =
        \begin{bmatrix}
          6 & 11 & 6
        \end{bmatrix}
      \end{equation}
      \begin{equation}
        \bar k =
        \begin{bmatrix}
          4 & 22 & 30
        \end{bmatrix}
      \end{equation}
      \begin{equation}
        Q = P^{-1} =
        \begin{bmatrix}
          0 & 0 & -2 \\
          0 & 1 & -4 \\
          1 & -4 & 13 \\
        \end{bmatrix}
        \begin{bmatrix}
          1 & 6 & 11 \\
          0 & 1 & 6 \\
          0 & 0 & 1 \\
        \end{bmatrix} =
        \begin{bmatrix}
          0 & 0 & -2 \\
          0 & 1 & 2 \\
          1 & 2 & 0
        \end{bmatrix}
      \end{equation}
      \begin{equation}
       P = Q^{-1} =
        \begin{bmatrix}
          0 & 0 & -2 \\
          0 & 1 & 2 \\
          1 & 2 & 0
        \end{bmatrix} =
        \begin{bmatrix}
          -2 & -2 & 1\\
          1 & 1 & 0 \\
          -\frac 1 2 & 0 & 0
        \end{bmatrix}
      \end{equation}
      \begin{equation}
        k = \bar k P = 
        \begin{bmatrix}
          4 & 22 & 30
        \end{bmatrix}
        \begin{bmatrix}
          -2 & -2 & 1\\
          1 & 1 & 0 \\
          -\frac 1 2 & 0 & 0
        \end{bmatrix} = 
        \begin{bmatrix}
          -1 & 14 & 4
        \end{bmatrix}
      \end{equation}
      \begin{equation}
        K =
        K_v +
        \bar v k =
        0 +
        \begin{bmatrix}
          0 \\
          1
        \end{bmatrix}
        \begin{bmatrix}
          -1 & 14 & 4
        \end{bmatrix} =
        \begin{bmatrix}
          0 & 0 & 0 \\
          -1 & 14 & 4
        \end{bmatrix}
      \end{equation}
    \end{enumerate}
  \end{enumerate}    