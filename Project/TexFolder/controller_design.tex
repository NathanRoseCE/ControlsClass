\subsection{The Controller}
To begin with, we will develop a controller that will be used in order to apply feedback and make the system
controllable.

{\LARGE \color{red} TODO: re-enable this}
% \ PY {non_linear_system | design_k }

First the goal is to see if the system behaves well in a non-linear system from various distances from the
origin.
In order to make sure none of the state space variable explode but also wanting to make sure that multiple
starting states can be overlaid on the same graph the following C matrix was used in a linear model
\begin{equation}
  C =
  \begin{bmatrix}
    1 & 1 & 1 & 1 & 1 & 1 \\
  \end{bmatrix}
\end{equation}

% \ PY {non_linear_system | eval_k_linear }

from these graphs, it can be seen that even non-sensically high values can be controlled down to zero, thanks
to the linear approximation and the controller being able to compensate for it. Howver this does show that the
controller works in at least this limited case.

Now to see the usefulness of the controller, let us re-apply the system to the non-linear system. The inital
states are adjusted by a factor of 100, which was necessary in order to make sure that all of the states
converged. THe same C matrix was used for a linear output

% \ PY {non_linear_system | eval_k_nonlinear }