\section{Introduction}

\subsection{Defining the non-linear system}
A doubly inverted pendulum is a pendulum on a pendulum. What makes this a classical control problem is the
inverted nature of the system. That is instead of hanging the pendulum, the goal is to balence it upright.

The equations of motion were derived with assistance from \cite{eq_of_motion}.
{\LARGE \color{red} TODO: re-add the filter}
% \ PY{ 'fix cite with a filter' | TODO }

This will not be a derivation of the work, as the work follows closely from \cite{eq_of_motion} which I used
as a source to help derive the non-linear equations of motion. 

% \ PY{ non_linear_system | nonlinear_equation_symbolic }
% \ PY{ non_linear_system | nonlinear_equation_evaluated }
% \ PY{ 'fix runnon equations...... somehow' | TODO }

\subsection{Defining the linear system}
The Linearization was simply a jacobian taken at 0,0,0 which is an equilibrium point that the system will be
controlled to.

% \ PY{ non_linear_system | linear_equation_symbolic }
% \ PY{ non_linear_system | linear_equation_evaluated }
